% Due to time constraints, we are not going to use the Excalibur
% Tutorial as our test case. Instead, we will do a few basic tasks to
% check if the machine is usable.

% We are going to download and install Remarkable for this purpose.
% Since downloading and installing software will be part of the tasks
% required to set up Excalibur, this seems like a good trade off.

% Further, to make this test case more adapted to a real life situation,
% we will be running one or two Youtube videos in the background as well
% as having a word editor opened while doing the installation.

% If we give our machine enough resources to work properly, then these
% tasks should be doable without any major problems.

We will evaluate the solution with regards to the non-functional
requirements to check how well these have been tackled. Admittedly,
the only part of the final solution that has not been solved yet is to
automatically ensure SSH and RDP\footnote{Remote Desktop Protocol is
used for sharing the Desktop of the instance} access once the instance
is created.  So this a step which the DevOps engineer needs to do
manually before being able to respectively provision the machine and
set up the Excalibur Environment.

Also, we will consider that the DevOps engineer uses this solution
more than once since the first time around there are quite a few
things to set up but after that you are good to go and can deploy new
instances quite fast. So for the evaluation we will not consider the
things you have to do only the very first time.

We will use the following scale ranging from handled the worst to best
for evaluation purposes: \verb|-|, \verb|--|, \verb|+|, \verb|++|.

\begin{enumerate}

	\item \textbf{Operability}.  In order to evaluate this we will
	take into consideration all the things the DevOps engineer must be
	able to do beforehand in order to tackle the given solution.

	For our solution, the DevOps engineer must at least know how to
	set up GCP and a project, how to set up the gcloud command line
	tool and how to get SSH access to the instances. After that, the
	scripts handle pretty much everything. Granted, the tutorial we
	provided gives step by step instructions for these tasks. Thus we
	conclude that the operability is \verb|+|.

	\item \textbf{Efficiency}.	In order to evaluate this
	non-functional requirement, we will look at the amount of steps
	necessary to acquire the Virtual Machine and get them ready for
	the Excalibur Environment. We will orient ourselves on the
	tutorial that will be presented along side the script.

	If you want to create the virtual machine manually it will take
	you, give or take, 3 steps. If you opt for the script or gcloud
	tool there is only 1 step. Although the gcloud tool might take you
	a few more steps to find the right things to put in the flags.

	For provisioning you have to set up SSH access manually and
	afterwards you need to set up RDP access as well. These two steps
	might be a little cumbersome. So the efficiency is \verb|+|.

	\item \textbf{Satisfaction}.  In order to evaluate this, we will
	again look at the amount of steps necessary but also at the time
	required to execute them.  This is due to the fact that more and
	lengthy steps to follow decrease the ease and speed at which you
	can repeat a given solution.

	Due to the two steps on SSH and RDP that need to be done manually
	and may that may be a little annoying, we will say that
	satisfaction is \verb|-|.

	\item \textbf{Maintainability}.  For evaluation of this
	non-functional requirement, we look at how difficult and how much
	effort it may take for the developer to to tweak the creation and
	provisioning.

	The two steps that require manualy intervention, namely the SSH
	and RDP setup, do not give rise to maintainability issues because
	you set them up once and there is nothing to worry about anymore.
	Added to the fact that everything is handled by scripts which you
	can go and edit, we conclude that maintainability is \verb|++|.

	\item \textbf{Reliability}.  For our evaluation purpose, we will
	try to look at how many things can go wrong during the Virtual
	Machine creation. So the fewer things there are that can go wrong,
	the more reliable the solution will be.

	Except for the two steps in setting up SSH and RDP nothing can
	really go wrong because the script handles the rest. For setting
	up SSH there are not too many things that can go wrong considering
	there are not a lot of steps and they are rather simple. However
	the RDP setup requires a few steps that may or may not demand a
	little more attention, since you need to have a look at one or the
	other configuration file. Hence we will say that reliability is
	\verb|+|.

\end{enumerate}

To summarize, we end up with table \ref{tab:nfr} which also compares
this BSP's solution with the previous semester's solution.

\begin{table}
	\begin{tabular}{l||cc}
		NFR              & GCP with Ansible & Vagrant with Ansible \\\hline
		Operability      &   \verb|+|       &    \verb|++| \\
		Efficiency       &   \verb|+|       &    \verb|++| \\
		Satisfaction     &   \verb|-|       &    \verb|++| \\
		Maintainability  &   \verb|++|      &    \verb|++| \\
		Reliability      &   \verb|+|       &    \verb|++| \\
	\end{tabular}
\end{table}
