As already mentioned, part of this deliverable relies on work done in
the previous semester. In the last semester, we also tried to produce
an automated solution, however we were working locally on our
machines. In this semester we tried to adapt the knowledge aquired and
bring it to the next level, namely cloud computing. 

While the basic concepts remain the same, we now have to look at the
subject from a different perspective which is due to the different
nature of working locally versus working in the cloud. Before, all the
DevOps could do was to provide scripts the user had to run himself.
We considered there was no way for a DevOps to interact with the
user's machine which means that actually creating the machine with the
provided solution was up to the user.

Now we want a DevOps to create and set up the machines in the cloud
and the Excalibur user will merely have to connect to it and is ready
to go. There is close to nothing that the user will have to do in
order to get access to an Excalibur Environment.  Essentially, what we
are trying to achieve is Software as a Service.

A first difficulty for this BSP was to create virtual machines in the
cloud. We had to get accustomed to the Google Cloud Platform and learn
how to use it. This was a relatively easy and straight forward process
because Google provided interactive tutorials to guide its users
through the basics of the platform.

After getting the hang on the basics, we immediately started wondering
if we could reuse the scripts that handled the provisioning from the
previous semester. We opted for a provisioning solution that was not
specific to the previous semester's solution but would allow for a
broader range of application.  A quick research revealed that indeed,
we were able to reuse the provisioning scripts. Only one minor
modification had to be done, namely \textit{how} do we execute
it\footnote{In the previous semester, the executing of these scripts
was very specific to the solution}.

Lastly, we inspected the \texttt{gcloud} command line tool which
basically allows us to interact with GCP through the terminal. We
wrote scripts around this tool to have new machines be created
automatically. Automating the provising also relies on some output
generated by this tool.
