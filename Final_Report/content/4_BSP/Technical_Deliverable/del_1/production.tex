The first thing we had to get comfortable with is the Google Cloud
Platform. There were a few things we needed to know and had to set up
before we could even get started with creating virtual machines.

A first requirement was to have a gmail account to be able to operate
on GCP. Next, we need to create a new project in which we manage and
manipulate our virtual machines.
\
\textcolor{purple}{You can read about it in the tutorial on
github\footnote{\url{https://github.com/niveK77pur/ExcaliburEnvironmentGCP}}.
Please Refer to section \textbf{Setting up GCP} and section
\textbf{Creating Virtual Machines}, subsection \textbf{Manual
Creation}.}
\
After having created a first virtual machine, with very few resources
attributed to it because it was merely a test and we did not want to
have high billing for this, we inspected ways to connect to it.
\
\textcolor{purple}{Please refer to section \textbf{Connecting to
Virtual Machines} and its subsections in the tutorial.}
\
The first obvious choice was to connect directly from GCP. However,
this did not allow us to provision the machine using our Ansible
scripts. So we were forced to inspect manual SSH connection. We had to
ensure SSH connection to the virtual machine in order to run our
Ansible provisioning. 

This is due to the fact, that Ansible relies on a host file or
inventory file as it is called. This file contains a list of hosts to
provision and Ansible will connect to these host via SSH. There seem
to be methods for connecting to a host on GCP from Ansible without
requiring manual SSH connection, but these methods looked rather
complex and we did not get them to work. That said, we stuck to
manually ensuring SSH connection.

\begin{verbatim}
	 [GCP]
	 35.227.50.0

	 [GCP:vars]
	 ansible_user = student002
\end{verbatim}

Here is an example of a host file that was written manually in order
to check if provisioning works. With the \verb|[...]| we can specify a
group of hosts. That way we can have Ansible provision only a specific
collection of machines if we so desire. However, in the Ansible
scripts, we have set it such that all hosts will be provisioned.

Furthermore, we have the \verb|[...:vars]| with which we can specify
variables that apply to a certain group of hosts. Here we say what
user to connect as on the remote machine. With the scripts that
automate the solution, our host file will be generated and look a
little different.

\begin{verbatim}
	 35.227.50.0	 ansible_user = student002
	 35.653.43.0	 ansible_user = other_user
\end{verbatim}

This way we are specifying the \textit{main} user for each machine.
Since each machine will have multiple users we need to specify a user,
with which to connect as, to do the provisioning. The multiple users
are required for ensuring Remote Desktop Protocol (RDP) connection.
Somehow it is not possible to connect to the remote machine's desktop
with the main user, so that is why there will be multiple users.

After being able to handle the provisioning, which is a big step in
the right direction, we looked into how to automate the creation of
the virtual machine. Our first attempt, was to integrate this into our
Ansible workflow. There are methods provided by Ansible to create
virtual machines on GCP but here again, we did not manage to get them
to work. So we went for the gcloud tool for doing this. The gcloud
command we used to create virtual machines is a follows.
\
\textcolor{purple}{Please refer to the tutorial section
\textbf{Creating Virtual Machines} subsection \textbf{Creation using
gcloud} for more details on this command.}

\begin{verbatim}
gcloud compute instances create "$1" \
    --machine-type="$MACHINE_TYPE" \
    --boot-disk-size="$BOOT_DISK_SIZE" \
    --image-project="$IMAGE_PROJECT" \
    --image-family="$IMAGE_FAMILY" \
    --metadata="$METADATA" 
\end{verbatim}

After having all of this figured out, we wrote one script around the
above command to handle the virtual machine creation and another
script is used to handle the provisioning. This second script relies
on gcloud to get the external IP of the remote machine because the IP
changes each time you start it anew. So we write the IP along with the
\textit{main user}, as explained above, to a new host file and have
ansible provision the machines specified in that host file.

To wrap things up, we wrote a third script that calls these two
scripts. That way we only need to execute one script and everything is
handled for us. The last script has the following content.

\begin{verbatim}
	 #!/bin/bash
	 
	 # $1 : instance name
	 # $2 : main user name at the instance
	 
	 bash ./new_vm.sh "$1"
	 bash ./provision.sh "$1" "$2"
\end{verbatim}
