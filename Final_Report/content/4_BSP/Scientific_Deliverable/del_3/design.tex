% {\color{gray}
% Provide the necessary and most useful explanations on how those deliverables have been produced.
% }

Since the very beginning of the BSP, we were thinking about the best
way to present the tutorial and convey its content. There were
basically two options we were considering. 

\begin{enumerate}
	 \item Video format. This would not be a bad decision because it
		is easy to follow. The person making the tutorial can
		precisely show how to do the various steps and what to click
		and so on. It would also be very easy to highlight potential
		issues and side information on the fly. In textual form if
		not done well, presenting such information can quickly make
		the tutorial lose focus from the topic at hand. Plus, there
		is a video presentation to do for the BSP, so that would
		allow me to practice recording a video and reuse some of the
		work done.

		However, making these videos takes quite a lot of time and
		preparation. In case of mistakes you would have to record a
		specific section again and maybe even multiple times if you
		repeatedly make mistakes. If you notice a problem a little
		further down the line, it might also be hard to fix. 

	\item Textual format. The main advantage here is that it is a lot
		faster to produce, since mistakes can be fixed with more ease
		and less effort. You can also rearrange and restructure the
		text without losing continuity because if you use
		transitions\footnote{i.e. saying "elaborated in the next
		part","as previously mentioned", ..."}, rearranging parts of
		the video may make it lose continuity while in textual form,
		you simply remove or change these transitions.

		In case you already know how to use the solution or have read
		the tutorial and you have some doubts later down the road, you
		can quickly refer to a specific section from the tutorial. In
		addtion, if it is well structured, the tutorial allows for a
		quick and easy read and you can provide things to be aware of
		and miscellaneous information without deviating from the main
		topic.

		Nonetheless, it is a little harder to visualize steps to be
		done. If we include images, for illustration purposes, that
		are not annotated or described well enough, they might lead to
		confusion\footnote{i.e. not understand where they belong or
		what they are meant to illustrate}.
\end{enumerate}
	
In both cases, you may also run the risk of not structuring your
tutorial well which makes it hard to read or watch and lose focus of
the main topic at hand.

Considering all this, we opted for a tutorial in textual form, mainly
due to its flexibility and faster production.
