% {\color{gray}
% Provide any objective elements to assess that your deliverables do or do not satisfy the requirements described above. 
% }

The assessment of the tutorial is made by providing it to a list of
people given by the BSP tutor. These people will respond a to a
survey.  The answers of that survey are used to determine the quality
of the tutorial in terms of suitability.

You can find screenshots of the results in the appendix section
\ref{app:survey}. We will use these to assess the tutorial. We were
only able to get two responses for the survey which admittedly is not
enough to make a final judgement but we will go with what we have.

From screenshot \ref{fig:survey1} we can see that the two people who
did the survey were familiar with programming, the linux shell and
virtual machines which are important factors in our context.
Furthermore, from screenshot \ref{fig:survey2}, we see that one of
these individuals has quite a few years of experience in his
respective fields.

Screenshots \ref{fig:survey3} and \ref{fig:survey4} show that the
basic procedures were relatively well handled thanks to the tutorials
instructions. However the second question in screenshot
\ref{fig:survey4} until \ref{fig:survey6} reveal that the procedures
specific to our technical solution were tougher and the individuals
did not succeed or only so with a lot of trouble.  This might be
attributed to the fact that the solution might not be as straight
forward to use as anticipated, or that the tutorial does not provide
enough details on these subjects. This would be a part that needs
further elaboration to increase the quality.

Thus we conclude that our tutorial is of above average quality, due to
the fact that basic concepts were well presented but the important
technical part seemingly less so.
