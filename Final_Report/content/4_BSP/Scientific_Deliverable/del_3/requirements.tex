% {\color{gray}
% Describe here all the properties that characterize the deliverables
% you produced. It should describe, for each main deliverable, what are
% the expected functional and non functional properties of the
% deliverables, who are the actors exploiting the deliverables. It is
% expected that you have at least one scientific deliverable (e.g.
% ``Scientific presentation of the Python programming language'',
% ``State of the art on quality models for human computer interaction'',
% \ldots.) and one technical deliverable (e.g. ``BSProSoft - A
% python/django web-site for IT job offers retrieval and analysis'',
% \ldots). 
% }

The final scientific deliverable for this project is a tutorial
targeted towards a DevOps engineer and is meant to help him understand
and use our solution. The tutorial should talk about everything that
you may need to understand how the solution works and to build upon it
independently.

The functional requirement for this tutorial is, as already said, to
help the DevOps engineer in fulfilling the main techincal solution of
this BSP, namely to get a new Excalibur Environment.

The non-functional requirements include:
\begin{enumerate}
	\item \textbf{Completeness}, which specifies how well the tutorial
		 explains the given facts. To phrase it in a negated manner:
		 will the reader still have major doubts after reading the
		 tutorial?
	\item \textbf{Easy to read}, determines the reading quality of
		the tutorial.  We want the tutorial to feel light which in
		turn allows for an easy reading experience.
\end{enumerate}
