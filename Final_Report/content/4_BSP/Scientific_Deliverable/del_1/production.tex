With the help of the tutorials, we were able to understand various
terms and concepts related to GCP. The most important ones which we
dealt with the most will be presented in the following.

\begin{enumerate}

	\item \textbf{Projects} is where we perform actions and do stuff
	within GCP. For our sole purpose of creating remote virtual
	machines, we needed to create a project in which to create those.
	Added to this, we can set project-wide settings that will affect a
	lot of things, including virtual machines, that are contained
	within it. For example, we have a project in which we set a
	\textit{default user} that should be used for SSH connections.
	That way we do not need to specify explicitly it for every
	instance. So you can consider projects as a sort of working
	environment.

	\item \textbf{Instance} is essentially the term used by GCP to
	denote a virtual machine instance. 

	\item \textbf{Billing} is an important part of GCP because the
	services are not provided for free. Contrary to a lot of services
	out there, we do not have to pay a fixed monthly subscription fee
	to use them. The way it works for cloud computing in general is
	that you pay for what you use. What this means is that if you have
	a bunch of instances, which obviously require resources to run,
	and they are \emph{not} running, or in other words: they are shut
	down, then you do not pay for them. If you have instances that are
	running, then the amount you pay mainly depends on the resources
	the instance uses. You may even get discounts in specific cases. 

	\item \textbf{IaaS} is a concept that has been touched upon in the
	tutorials along side the other service models. As we have seen
	before, these are part of cloud computing. IaaS is important here
	because using GCP means that we rely on such a service.
	Essentially, Google provides us with the hardware and some
	software that comes with a chosen OS, so in other words the
	infrastructure, which we get to decide how to use.
	 
\end{enumerate}
