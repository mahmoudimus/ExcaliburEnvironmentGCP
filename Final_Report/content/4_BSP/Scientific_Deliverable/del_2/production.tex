SSH connections rely on so called \textit{key pairs}. This key pair is
composed of a private and a public key. To create this key pair we
simply had to run the following command \verb|ssh-keygen|. We will
then be prompted to name the file and to give a password with which to
encrypt them. If we do not keep the default name then we have to
specify where the private key is each time we want to connect through
SSH. If we set a password, we will be asked to enter it each time we
want to connect to an instance using the password protected key.

Once we have the keys, we are half way there. We will notice that two
keys were generated. One of them is the private and the other one is
the public key. The latter one has \verb|.pub| attached to the end.
The last step is to put the public key onto the remote machine inside
its \verb|$HOME/.ssh| folder and we are good to go. GCP also has a
feature that allows you to embed the public keys within the project.
That way you do not need to put them manually on the instances.

We should note that the private key is not meant to be shared. To
understand how it works, you can think of it as a real physical key
you use to unlock the door to your house. In this scenario the public
key is the door lock. You would not want to share your key with
everyone because you might end up having unwanted guests at your
place. The same applies to the private key.  Anyone with your private
key can access machines on which you have put your corresponding
public key. We want to avoid this as it can lead to security issues.
