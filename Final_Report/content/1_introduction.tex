% {\color{gray}
%
% This paper presents the bachelor semester project made by Motivated
% Student together with Motivated Tutor as his motivated tutor.  It
% presents the scientific and technical dimensions of the work done.
% All the words written here have been newly created by the authors
% and if some sequence of words or any graphic information created by
% others are included then it is explicitly indicated the original
% reference to the work reused. 
%
% This report separates explicitly the scientific work from the
% technical one. In deed each BSP must cover those two dimensions with
% a constrained balance  (cf. \cite{bics-bsp-reference-document}).
% Thus it is up to the Motivated Tutor and Motivated Student to ensure
% that the deliverables belonging to each dimension are clearly
% stated. As an example, a project whose title would be ``A multi-user
% game for multi-touch devices'' could define as
% scientific~\cite{armstrong2017guidelinesforscience} deliverables the
% following ones: \begin{itemize} \item Study of concurrency models
% and their implementation \item Study of ergonomics in human-computer
% interaction \end{itemize}
%
% The length of the report should be from 6000 to 8000 words excluding
% images and annexes.
%
% }

Over the course of this BSP, we are trying to achieve an automated
solution that eases the creation and setup of a new environment, more
specifically the Excalibur Environment.

To achieve our solution, we worked with the Google Cloud Platform
(GCP) to create our instances. Later on we found ways to automate the
creation of new instances on GCP. 

In order to have all the tools and packages needed for the Excalibur
Environment as well as for ensuring remote desktop access we used
\textit{Ansible} which allowed us to provision our instances. That
way, all the required stuff will be put on the machine automatically.

While trying to provision our machines however, we encountered an
issue that required us to look into how to set up SSH connections.

Finally, we also produced a lengthy tutorial that explains all the
things you need to know. Topics range from setting up GCP all the way
to some insight on the tools used during the BSP.
