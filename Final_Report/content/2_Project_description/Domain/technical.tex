Virtualization allows us to move from physical mediums to virtual ones
without really losing anything. The only major difference is that now
we do not manipulate the hardware directly but instead manipulate the
virtual hardware through specific software. This allows for great
flexibility in a lot of ways.

Along with virtualization, we dive into a concept called
\textit{Infrastructure as Code} which essentially means that we can
describe infrastructure\footnote{Infrastructure refers to the hardware
and software that compose a machine} through code or configuration
files. For example, we can write a little piece of code that creates a
virtual machine and inside that code we can give specifications for the
machine, like disk and memory size.

Furthermore, the concept of \textit{provisioning} will come into play
when talking about Infrastructure as Code. Provisioning allows us to
further specify characteristics of a machine. With this, we can have
various tasks be performed automatically. When calling this
provisioning while creating the virtual machine, we can for example
specify packages that should be installed right away or other
operations that should be made to obtain the machine we want.

Lastly, the concept of provisioning brings us to a tool called
\textit{Ansible}. This is a tool specifically designed for
provisioning machines with great ease. With Ansible, we can write
playbooks that basically say what should be done in order to provision
the machine. The language for these playbooks is \textit{yaml}, which
is very easy to use. This way we end up with a bunch of
\textit{configuration files} that describe that machine.
