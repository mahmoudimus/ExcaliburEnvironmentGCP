Cloud computing brings the domain of virtualization, as explained in
the next section, to yet another level. We are still manipulating a
virtual medium, but the virtual machine does not run locally on our
physical machines. Instead these machines run on servers somewhere and
we can controll them through a remote connection. This leads us to the
various service models employed by the cloud.

The different service models are Infrastructure as a Service (IaaS),
Platform as a Service (PaaS) and Software as a Service (SaaS). The
three service models are briefly explained in the following.
\begin{description}

	\item[IaaS] We are provided with the necessary infrastructure to
	run a virtual machine and the rest is up to us.  This is the model
	used by Google Cloud Platform.

	\item[PaaS] We get the necessary resource as well as an OS
	installed with some basic software. We can use the provided platform
	and modify it as we please. However we are not able to touch the
	underlying virtual hardware.

	\item[SaaS] Usually SaaS is accessed through a web browser and we
	are provided with everything needed to run a specific software. We
	are only able to access the software and none of its underlying
	components, like the OS or the hardware.

\end{description}
